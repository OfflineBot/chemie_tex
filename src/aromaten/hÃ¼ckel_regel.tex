\section{Hückel-Regel}
Ein aromatischer Zustand liegt vor, 
wenn die Moleküle einer Verbdindung eine ebene Ringstruktur mit einem ringförmigen geschlossnen 
$\pi$-Elektronensystem mit ($4n + 2$) delokalisierten $\pi$-Elektronen aufweisen 
($n = 0,1,2,..$).

Das heißt ein Ring mit $6$ $C$ passt, weil $4n + 2 = 6$ ist möglich ($4*1+2 = 6$).

$8$ $C$ würde nicht passen, weil $4n + 2 \neq 8$ und $n$ darf keine Kommazahl sein.

