\section{Begriffe}
\subsection{Mesomerie Energie} \label{sec:mesomerie}
Verbindungen mit delokalisierten Elektronen sind in der 
Regel Energie ärmer und somit stabiler. 
Die Energiedifferenz zwischen den tatsächlich existierenden 
Teilchen und den hypothetischen Teilchen bezeichnet man als 
Mesomerie Energie.

\subsection{Benzol - Benzen}
Früher Benozol, heute Benzen. (beides wird benutzt)
\begin{multicols}{2}
\begin{itemize}
    \item $C_6H_6$
    \item Cyclohexa-1,3,5-trien
    \item Molare Masse: 78,14
    \item unlöslich in wasser /$\rightarrow$ organisches Löslich
    \item Einatmen sehr gefährlich (Krebserregend)
    \item Farblos, aromatisch riechend
    \item Schmelztemperatur: 5,5°C
    \item Siedetemperatur: 80,1°C
\end{itemize}
\end{multicols}

\subsection{Heterocyclische Aromaten} 
\label{sec:heteroaromaten}
Heterocyclische Aromaten sind Aromaten die in ihrem Ring mindestens ein \textbf{nicht} $C$-Atom haben.

\textbf{Beispiel}: (Pyridin. Hat ein $N$ anstelle eines $C$-Atoms) 

\chemfig{ C-[:60]C=[:120]N-[:180]C=[:240]C-[:300]C=[:0,0.85] } 


\subsection{Polycyclische Aromaten} 
\label{sec:polyaromaten}
Polycyclische Aromaten sind Aromaten die mit weitern benzenartigen Aromaten verbunden sind und somit sich die $\pi$-Bindungen über den Ring verteilen.

\textbf{Beispiel}: (Naphthalin)

\chemfig{ 
C-[:60]C
(=C-[:300]C=[:240]C-[:180]C=[:120,0.85])
-[:120]C=[:180]C-[:240]C=[:300]C-[:0,0.85] } 
