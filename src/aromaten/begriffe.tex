\section{Begriffe}
\subsection{Mesomerie Energie}
Verbindungen mit delokalisierten Elektronen sind in der 
Regel Energie ärmer und somit stabiler. 
Die Energiedifferenz zwischen den tatsächlich existierenden 
Teilchen und den hypothetischen Teilchen bezeichnet man als 
Mesomerie Energie.

\subsection{Benzol - Benzen}
Früher Benozol, heute Benzen. (beides wird benutzt)
\begin{itemize}
    \item $C_6H_6$
    \item Cyclohexa-1,3,5-trien
    \item Molare Masse: 78,14
    \item unlöslich in wasser /$\rightarrow$ organisches Löslich
    \item Einatmen sehr gefährlich (Krebserregend)
    \item Farblos, aromatisch riechend
    \item Schmelztemperatur: 5,5°C
    \item Siedetemperatur: 80,1°C
\end{itemize}

\subsection{Phenol}
\begin{itemize}
    \item Reagiert sauer
    \item Besitzt eine Hydroxy-Gruppe
\end{itemize}

\subsection{Aminobenzol - (Anilin)}
\begin{itemize}
    \item Reagiert basisch (schwach)
    \item Besitzt eine Amino-Gruppe
\end{itemize}

\subsection{Heterocyclische Aromaten}

\subsection{Polycyclische Aromaten}
