\section{Benzolderivate}
\label{sec:benzolderivate}

\begin{multicols}{2}

\textbf{Phenol} ($Benzen + OH$)

\chemfig{ C-[:60]C=[:120]C
    (-[:45]OH)
    -[:180]C=[:240]C-[:300]C=[:0,0.85] } 


\textbf{Toluol} ($Benzen + CH_3$)

\chemfig{ C-[:60]C=[:120]C
    (-[:45]CH_3)
    -[:180]C=[:240]C-[:300]C=[:0,0.85] } 

\end{multicols}
\begin{multicols}{2}

\textbf{Benzaldehyd} ($Benzen + CHO$)

\chemfig{ C-[:60]C=[:120]C
    (-[:45]C(=[2]O)(-[:-33]H))
    -[:180]C=[:240]C-[:300]C=[:0,0.85] } 

\textbf{Benzosäure} ($Benzen + COOH$)

\chemfig{ C-[:60]C=[:120]C
    (-[:45]C(=[2]O)(-[:-33]OH))
    -[:180]C=[:240]C-[:300]C=[:0,0.85] } 

\end{multicols}
\begin{multicols}{2}

\textbf{Styrol} ($Benzen + CCH_2$)

\chemfig{ C-[:60]C=[:120]C
    (-[:45]C(=CH_2))
    -[:180]C=[:240]C-[:300]C=[:0,0.85] } 


\textbf{Phenylalanin} ($Benzen + C_2NH_2(COOH)$)

\scalebox{0.8}{
\chemfig{ C-[:60]C=[:120]C
    (-[:45]C-C(-[6]NH_2)-C(=[2]O)-OH)
    -[:180]C=[:240]C-[:300]C=[:0,0.85] } 
}

\end{multicols}

