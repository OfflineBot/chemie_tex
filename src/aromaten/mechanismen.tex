\section{Mechanismen}

\subsection{Radikale Substitution}
Eine radikale Substitution ist eine chemische Reaktion, bei der ein Atom oder eine Atomgruppe in einem Molekül durch ein anderes Atom oder eine andere Atomgruppe ersetzt wird, wobei Radikale (reaktive Teilchen mit ungepaarten Elektronen) als Zwischenprodukte gebildet werden.

\textbf{Startreaktion}: \\
Ein Modelül (Hier $Br_2$) wird geteilt um als radikal zu agieren.

\chemfig{
    Br-Br
}
\hspace{0.3cm}
$\xrightarrow{\text{Licht oder Wärme}}$
\hspace{0.3cm}
\chemfig{
    Br\cdot
}
+
\chemfig{
    \cdot Br
}

\vspace{0.5cm}

\textbf{Reaktionskette}: \\
Öffnung des Moleküls durch Radikal. 

\chemfig{
    C(-[0]H)(-[2]H)(-[4]H)(-[6]*6(=-=-=-))
} 
\hspace{0.3cm}
+
\hspace{0.3cm}
\chemfig{
    \cdot Br
}
\hspace{0.3cm}
$\longrightarrow$
\hspace{0.3cm}
\fbox{
    \color{red}
    \chemfig{
        C\cdot(-[2]H)(-[4]H)(-[6]*6(=-=-=-))
    } 
}
\hspace{0.3cm}
+
\hspace{0.3cm}
\chemfig{
    H-Br
}

\fbox{
    \color{red}
    \chemfig{
        C\cdot(-[2]H)(-[4]H)(-[6]*6(=-=-=-))
    } 
}
\hspace{0.3cm}
+
\hspace{0.3cm}
\chemfig{
    Br-Br
}
\hspace{0.3cm}
$\longrightarrow$
\hspace{0.3cm}
\fbox{
    \color{cyan}
    \chemfig{
        C(-[0]Br)(-[2]H)(-[4]H)(-[6]*6(=-=-=-))
    }
}
\hspace{0.3cm}
+
\hspace{0.3cm}
\chemfig{
    \cdot Br
}

\vspace{0.5cm}

\textbf{Abbruchreaktion}: \\
Es gibt 3 verschiedene wege wie die Reaktion beendet werden kann.

1) Das Radikal wird wieder zusammengefügt \\
\chemfig{
    Br\cdot
}
\hspace{0.3cm}
+
\hspace{0.3cm}
\chemfig{
    \cdot Br
}
\hspace{0.3cm}
$\longrightarrow$
\hspace{0.3cm}
\chemfig{
    Br-Br
}

2) Ein zweites Molekül schließt sich dem ersten an (kann in weitere Kette reagieren) \\
\fbox{
    \color{cyan}
    \chemfig{
        C(-[0]Br)(-[2]H)(-[4]H)(-[6]*6(=-=-=-))
    }
}
\hspace{0.3cm}
+
\hspace{0.3cm}
\chemfig{
    C(-[0]H)(-[2]H)(-[4]H)(-[6]*6(=-=-=-))
}
\hspace{0.3cm}
$\longrightarrow$
\hspace{0.3cm}
\chemfig{
    C(-[2]H)(-[4]H)(-[6]*6(=-=-=-))
    (-[0,2.0]C(-[2]H)(-[0]H)(-[6]*6(=-=-=-)))
}
\hspace{0.3cm}
+
\hspace{0.3cm}
\chemfig{
    H-Br
}

3) Ein Radikal schließt sich dem Molekül an und es wird nicht weiter reagiert \\
\chemfig{
    C\cdot(-[2]H)(-[4]H)(-[6]*6(=-=-=-))
} 
\hspace{0.3cm}
+
\hspace{0.3cm}
\chemfig{
    Br
}
\hspace{0.3cm}
$\longrightarrow$
\hspace{0.3cm}
\chemfig{
    C(-[0]Br)(-[2]H)(-[4]H)(-[6]*6(=-=-=-))
}


\subsection{Elektrophile Substitution}
Eine elektrophile Substitution ist eine chemische Reaktion, bei der ein Elektronenpaardonor (Nukleophil) ein Atom oder eine Atomgruppe in einem Molekül durch ein Elektrophil ersetzt, das ein Elektronenpaar anzieht und dabei eine Zwischenverbindung bildet.



\subsection{Elektrophile Addition}

