\section{Strukturaufklärung}
\textbf{Nicht Abi relevant!!} \\
1925 Michael Faraday entdeckt Benzol in Leuchtglasflaschen. \\
Er ermittelte das Massenverhältnis $C:H = 12:1$. \\
Verhältnisformel: $C_1:H_1$ \\
$\rightarrow C_6H_6$ \\

1865 - 1872 Kekulê: 6-C-Ring von \\
\chemfig{ C-[:60]C=[:120]C-[:180]C=[:240]C-[:300]C=[:0,0.85] } \\
1872: Zwei Gleichwertige die sich nur in der Lage von Doppelbindungen Unterscheiden und ständig den Platz wechseln können. \\
Struktur:
\begin{multicols}{2}
\begin{itemize}
    \item planar
    \item Bindung
    \item Bindungswinkel: $120$°
    \item Bindungslänge: $139$pm \\
        Einfachbindung: $154$pm \\
        Doppelbindung: $134$pm
    \item $\sigma$-Bindungsgerüst mit delokalisierten $\pi$-Elektronen
\end{itemize}
\end{multicols}

