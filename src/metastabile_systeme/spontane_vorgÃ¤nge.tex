\section{Spontane Vorgänge} \label{sec:spontane_vorgaenge}
Chemische Reaktionen, die energetisch betrachtet bei einer gegebenen Temperatur ohne äußere Einflüsse ablaufen können.
Die Geschwindigkeit spielt dabei keine Rolle.

\subsection{Gibbs Hemlholz Gleichung}
Die freie Enthalpie $G$ beschreibt spontane Reaktionen. \newline
$G = H - T * S$ 
$\Delta G = \Delta H - T * \Delta S$ 
\begin{itemize}
    \item $\Delta _rH$ = Reaktionswärme $Q_r$ bei konstantem Druck
    \item $T * \Delta S$ = Temperatur mal die Entropieänderung
    \item $\Delta_rG < 0$ = Reaktion ist Exergonisch, läuft freiwillig ab
    \item $\Delta_rG > 0$ = Reaktion ist Endergonisch, kann durch Energieerzeugung von außen erzwungen werden
\end{itemize}
\

\subsection{Ordnung im System}
Befinden sich Teilchen in einem begrenzten Bereich an bestimmten Plätzen, 
bezeichnet man diesen Zustand als geordnet. 
Die Orndung wird geringer,
wenn für die Teilchen der Raum und die Anzahl der möglichen Plätze gröber werden.
Dies beruht auf der Ausbreitung der Energie. \newline
Vorgänge mit geringer Ordnung:
\begin{itemize}
    \item Diffusion
    \item Rost
    \item Schmelzen
\end{itemize}

Die Ordnung nimmt ab wenn...
\begin{itemize}
    \item der Raum größer wird
    \item die Geschwindigkeit der Teilchen zu nimmt (Temperatur)
    \item Anzahl der Teilchen zu nimmt
    \item Temperatur, Konzentration oder Druck gleichen sich aus
\end{itemize}

Man muss die Ordnung sowohl im System als auch in der Umgebung betrachten. 
\hyperref[sec:entropie]{Entropie} ist ein Maß der Unordnung.
