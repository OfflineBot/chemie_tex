\section{Homologe Reihen}
Beispiele sind jeweils mit Ethan- gewählt.
\begin{enumerate}
    \item Alkane: $C-C-$Einfachbindung \\
        \chemfig{H-C(-[2]H)(-[6]H)-C(-[2]H)(-[6]H)-H}

    \item Alkene: $C=C-$Doppelbindung \\
        \chemfig{H-C(-[2]H)(-[6]H)=C(-[2]H)(-[6]H)-H}

    \item Alkine: $C\equiv C-$Dreifachbindung \\
        Wie Alkene/Alkane nur mit 3 Bindungen

    \item Alkohole: $R-OH$ \\
        \chemfig{C(-[2]OH)(-[4]H)(-[6]H)-C(-[2]H)(-[6]H)-H}

    \item Aldehyde: $R-CHO$ (Doppelbindung von $C=O$) \\
        \chemfig{C(=[3]O)(-[5]H)-C(-[2]H)(-[6]H)-H}

    \item Ketone: $R-CO-R'$ (Doppelbindung von $C=O$)

    \item Carbonsäuren: $R-COOH$ (Eine Doppelbindung $C=O$ und eine $O-H$)

    \item Ester: $R-COO-R'$ (Eine Doppelbindung $C=O$ sonst $R-C-O-R'$)

    \item Amine: $R-NH_2$

    \item Amide: $R-CONH_2$

    \item Nitroverbindung: $R-NO_2$

    \item Halogenalkane: $R-X$, wobei $X$ ein Halogen ist: $F$, $Cl$, $B4$, $I$

\end{enumerate}

