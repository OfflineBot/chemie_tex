\section{Oxidation} \label{sec:oxidation}
Oxidation ist ein chemischer Prozess, bei dem ein Element oder eine Verbdindung Elektronen verliert,
während es mit einem anderen Stoff reagiert. 
Die Substanz, die Elektronen verliert,
wird als Reduktionsmittel bezeichnet, während die Substanz, 
die Elektronen gewinnt, als Oxidationsmittel bekannt ist.
Ein Beispiel für eine Oxidationsreaktion ist die Reaktion von Eisen mit Sauerstoff, bei der Eisenoxid ensteht:\\ \ \\
$4Fe + 3O_2$ \textrightarrow\ $2Fe_2O_3$ \\ \ \\
In diesem Beispiel verliert das Eisen Elektronen und wird oxidiert, während der Sauerstoff Elektronen gewinnt und reduziert wird.
