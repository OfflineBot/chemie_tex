
\section{Katalysator} \label{sec:katalysator}
Ein Katalysator ist ein Stoff um die \hyperref[sec:aktivierungsenergie]{Aktivierungsenergie} herabzusetzen.  \\
\textcolor{red}{"Der Katalysator geht unverändert aus der Reaktion heraus."} \\
Er bietet der Reatkion eine andere Route wodurch es einfacher wird zu Reagieren.\\
Arten es Katalysator: \\
\begin{itemize}
    \item Homogene Katalyse: 
        Der Katalysator und der Reaktionspartner sind beide in der gleichen Phase. 
        Sprich, beide gasförmig oder flüssig. 
        Dadurch können sich die Stoffe gut vermischen.
    \item Heterogene Katalyse: 
        Beide Stoffe liegen in verschiedenen Phasen vor. 
        Der Katalysator ist im festen Zustand während die Edukte flüssig oder gasförmig sind.
    \item Biokatalyse:
        Katalyse die im Körper statt findet (biologische Systeme)
    \item Autokatalyse: 
        Bei der Reaktion entsteht der Katalysator selbstständig, daher braucht man keinen extra hinzufügen.
\end{itemize}

