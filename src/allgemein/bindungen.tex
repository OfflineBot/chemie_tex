
\section{Bindungen}

\subsection{Elektronenpaarbindungen}
\begin{itemize}
    \item Atome teilen sich die Elektronen
    \item Immer zwei Elektronen bilden eine Elektronenpaarbindung (polar, unpolar $\Delta$EN)
    \item bindende/nichtbindende Elektronenpaare
    \item Oktettregel/Edelgaskonfiguration soll erfüllt sein
    \item Einfach-, Zweifach-, Dreifachbindungen möglich
\end{itemize}

\subsection{Atombindung}

\subsection{Ionenbindung}
\begin{itemize}
    \item Salze
    \item Nichtmetall mit Metall
    \item Elektronen werden vollständig übertragen bzw. entzogen $\Delta$EN $>$ 1.5 \\
        \textrightarrow\ Kationen (positiv), Anionen (Negativ)
    \item Bohr'sche Atommodell \textrightarrow\ volle äußerste Schale
    \item positive und negative Ladungen ziehen sich elektrostatisch an
    \item Gitterstrukturen
    \item Wechselwirkungen in alle Raumrichtungen
    \item Hohe Schmelz und Siedepunkte
\end{itemize}

\subsection{Matallbindung}
\begin{itemize}
    \item Metall
    \item "Elektronengasmodell"
    \item Leitfähigkeiten erklärbar
    \item Duktilität (Verformbarkeit)
    \item chemische Bindung, die durch Anziehung zwischen Atomrümpfen und den freien Elektronen (Elektronengas) entsteht
\end{itemize}

\subsection{Koordinative Bindungen}
\begin{itemize}
    \item Kommen bei Komplexverbindungen vor
    \item Wechselwirkungen zwischen Zentralatom/ion und Liganden
    \item Spezielle Art der Elektronenpaarbindung
    \item Koordinationszahl als Anzahl der koordinativen Bindungen am Zentralteilchen
    \item Lewis Säure/Base-Theorie
\end{itemize}
Bsp: Diamminsilber(I)-Ions bei Tollens Reaktion, Tetraaquakupfer(II)-Ions bei Fehling Reaktion
