
\section{Redox Reaktion}

\subsection{Oxidation} 
\label{sec:oxidation}
Oxidation ist ein chemischer Prozess, 
bei dem ein Element oder eine Verbindung Elektronen verliert, 
während es mit einem anderen Stoff reagiert. 
Die Substanz, die Elektronen verliert, wird als Reduktionsmittel bezeichnet, 
während die Substanz, die Elektronen gewinnt, als Oxidationsmittel bekannt ist. 
Ein Beispiel für eine Oxidationsreaktion ist die Reaktion von Eisen mit Sauerstoff, 
bei der Eisenoxid entsteht:

$4Fe + 3O_2 \rightarrow 2Fe_2O_3$

In diesem Beispiel verliert das Eisen Elektronen und wird oxidiert, 
während der Sauerstoff Elektronen gewinnt und reduziert wird.

\subsection{Reduktion}
\label{sec:reduktion}
Reduktion ist das Gegenteil der Oxidation und bezeichnet einen chemischen Prozess, 
bei dem ein Element oder eine Verbindung Elektronen gewinnt. 
Die Substanz, die Elektronen gewinnt, wird als Oxidationsmittel bezeichnet, 
während die Substanz, die Elektronen verliert, als Reduktionsmittel bekannt ist. 
Ein typisches Beispiel für eine Reduktionsreaktion ist die Reaktion von Kupfer(II)-oxid mit Wasserstoff, 
bei der Kupfer und Wasser entstehen:

$CuO + H_2 \rightarrow Cu + H_2O$

In diesem Beispiel gewinnt das Kupfer(II)-oxid Elektronen und wird reduziert, 
während der Wasserstoff Elektronen verliert und oxidiert wird.

\subsection{Redox Reaktion}
\label{sec:redox}
Redoxreaktionen (Reduktions-Oxidationsreaktionen) sind chemische Reaktionen, bei denen Elektronen zwischen zwei Substanzen übertragen werden. Sie bestehen aus zwei Halbreaktionen: einer Oxidation, bei der Elektronen abgegeben werden, und einer Reduktion, bei der Elektronen aufgenommen werden. Ein Beispiel für eine Redoxreaktion ist die Reaktion von Zink mit Kupfer(II)-sulfat:

$Zn + CuSO_4 \rightarrow ZnSO_4 + Cu$

In dieser Reaktion gibt Zink Elektronen ab und wird zu Zinkionen oxidiert, während Kupfer(II)-ionen Elektronen aufnehmen und zu elementarem Kupfer reduziert werden. Hierbei fungiert Zink als Reduktionsmittel und Kupfer(II)-sulfat als Oxidationsmittel.

Oxidation: $Zn \rightarrow Zn^{2+} + 2e^{-}$
Reduktion: $Cu^{2+} + 2e^{-} \rightarrow Cu$
