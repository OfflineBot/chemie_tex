\section{Enthalpie} \label{sec:enthalpie}

\subsection{Standardreaktionsenthalpie}
Um Reaktionsenthalpien besser vergleichen zu können hat man die molare Standardreaktionsenthalpie eingeführt: $\Delta_RH°$ \\
Standardbedingungen = \textcolor{red}{25°C bzw. 298K | 1013mBar} \\
Es gibt verschiedene Standardenethalpien, besonders für Berechnungen geeignet sind:
\begin{itemize}
    \item molare Schemlz- und Verdampfungsenthalpie
    \item molare Verbrennungsenthalpie
    \item molare Bindungsenthalpie
    \item molare Bildungsenthalpie
\end{itemize}
Für energetische Betrachtungen kann es geschickt sein die Reaktion in gedachte Teilschritte zu zerlegen. \\
Satz von Hess 1840: Gesetz der konstanten Wärmesummen: \\ 
\textbf{"Die Enthalpieänderung von zwei Zuständen ist unabhängig vom Reatkionsweg"}

\subsection{Reaktionsenthalpie berechnen}
$V_aA + V_bB$ \textrightarrow\ $V_cC + V_dD$ \\
$V_x$ = stöchiometrischer Koeffizient \\
$\Delta_RH = \Sigma H_{Produkte} - \Sigma H_{Edukte}$

