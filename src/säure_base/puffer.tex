\section{Puffer} \label{sec:puffer}
Puffersysteme werden benutzt um den \hyperref[sec:ph_wert]{pH Wert} auf einem konstanten Wert zu halten.
Dabei gibt man eine schwache Säure und ihre konjugierte Base im Verhältnis von 1:1. \\
Pufferwirkung ist am optimalsten: $pH = pK_s +/- 1$ \\
Pufferkapaziät:
\begin{itemize}
    \item Stoffmenge an $H_3O^+$ die hinzugegeben werden muss, um den pH Wert um 1 zu verringern.
    \item Stoffmenge an $OH^-$ die hinzugegeben werden muss, um den pH Wert um 1 zu vergrößern.
\end{itemize}

\subsection{Puffer Berechnung}
\hyperref[sec:ph_wert]{pH Wert} = $pK_s + lg(\frac{|A^-|}{|HA|})$

