\section{Batterie und Akkumulatoren}


\textbf{Primärelemente}: 
\begin{itemize}
    \item Batterien
    \item Galvanische Elemente, bei dem die Elektrodenreaktion nicht umkehrbar
        ist 
    \item nicht wieder aufladbar \\
        $\implies$ wenn das Metall sich komplett zersetzt (Ionisierung)
\end{itemize}

\vspace{1cm}

\textbf{Sekundärelemente}: 
\begin{itemize}
    \item Akku
    \item elektrochemische Vorgänge innerhalb eines Sekundärelements sind umkehrbar
    \item Laden: elektrische Chemie $\rightarrow$ chemische Energie
    \item Entladen: chemische Energie $\rightarrow$ elektrische Energie
\end{itemize}

\vspace{1cm}

\textbf{Brennstoffzellen}:


\subsection{Batteriemodelle}

\textbf{Mangan Zink Batterie}:
\begin{itemize}
    \item Allgemein: 
    \item Anode: $2MnO_2(s) + 2H_2O(l) + 2e^- \rightarrow 2MnOOH(s) + 2OH^-(aq)$
    \item Kathode: $Zn(s) + 4OH^-(aq) \rightarrow [Zn(OH)_4]^{2-}(aq) + 2e^-$
    \item Redoxreaktion: $Zn(s) + 2MnO_2(s) + 2OH^-(aq) + 2H_2O(l) \rightarrow
        [Zn(OH)_4]^{2-}(aq) + 2MnOOH(s)$
    \item Zellspannung: $1.5V$
    \item Elektrolyt: Zinkpaste mit Kalilauge (KOH)
    \item Verwendung/Sonstiges: Fotoapparate, in vielen Geräten in Reihe
        geschaltet
\end{itemize}

\subsection{Brennstoffzelle}
\begin{itemize}
    \item Brennstoffzellen sind galvanische Zellen
    \item Es muss kontinuierlich Brennstoff und Oxidationsmittel hinzugefügt
        werden
    \item Direkte Umwandlung chemischer Energie in elektrische Energie
    \item Keine Emission von $CO_2$ $\rightarrow$ wird Wasserstoff durch
        Elektrolyse mit Naturstrom erzeugt
    \item Kalte Verbrennung
    \item Eine Zelle $1.23V$ (Theorie); Praxis $0.6 - 0.9V$ $\rightarrow$ daher
        Stacks
    \item Wirkungsgrad $=$ Nutzbare Energie$/$eingsetzte Energie Theorie $70-100\%$;
        Praxis: $40-70\%$
\end{itemize}
