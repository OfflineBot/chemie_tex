\section{Einteilung der Kunststoffe in verschiedene Arten}

\begin{itemize}
    \item \textbf{Sorte}: wie PE, PP, PET, ...
    \item \textbf{mechanischer/thermischer Eigenschaft}: Thermoplasten, Duromere, Elastomere
    \item \textbf{Reaktionstype}: Polykondensationsreaktion, Polymerisationsreaktion, Polyadditionsreaktion
\end{itemize}

\subsection{mechanischen/thermischen Eigenschaften}

\vspace{0.3cm}
\textcolor{orange}{orange} = Wird im Abi gerne abgefragt
\vspace{0.3cm}

\begin{minipage}[t]{0.4\textwidth}
    \textbf{Thermoplasten}: \textcolor{orange}{erhalten eine Doppelbindung oder zwei funktionelle Gruppen im Molekül}
    \begin{itemize}
        \item \textbf{Struktur}: Lineare anordnung (Faserartig)
        \item \textbf{Kräfte}: London-WW, \\ Wasserstoffbrücken-WW., \\ machmal London-WW
        \item \textbf{Eigenschaften}:
        \par Erweichen bei moderater Erwärmung, plastische Verformung möglich, brennbar, werden bei Kälte spröde, recyclebar, sprizgußverfahren
        \item \textbf{Beispiel}: PP, PE, PVE
    \end{itemize}

\end{minipage}
\hspace{0.5cm}
\vrule width 0.4pt
\hspace{0.5cm}
\begin{minipage}[t]{0.45\textwidth}

    \textbf{Duroplasten}: \textcolor{orange}{enthalten zwei Doppelbindungen oder drei funktionelle Gruppen im Molekül}
    \begin{itemize}
        \item \textbf{Struktur}: stark verzweigt, engmaschig
        \item \textbf{Kräfte}: kovalente Verbindungen
        \item \textbf{Eigenschaften}:
        \par zersetzen sich bei starker Erwärmung, keine plastische Verformung möglich, hart und spröde, nicht brennbar, Form entsteht bei Reaktion
        \item \textbf{Beispiel}: Harze
    \end{itemize}

\end{minipage}

\hrule
\vspace{0.3cm}

\textbf{Elastomere}: \textcolor{orange}{enthalten zwei oder drei funktionelle Gruppen im Molekül}
\begin{itemize}
    \item \textbf{Strukur}: Weitmaschig, weniger verzweigt als Duroplasten
    \item \textbf{Kräfte}:
    \item \textbf{Eigenschaften}: 
        \begin{itemize}
            \item schrumpfen bei moderater Erwärmung
            \item zersetzen sich bei starker Erwärmung
            \item keine plastische Verformung möglich
            \item gummielastische Verformung bei Krafteinwirkung
        \end{itemize}
    \item \textbf{Beispiel}: ABR, SBR, PU - Polyurethan
\end{itemize}


\subsection{Reaktionstypen}
\textbf{Polykondensationsreaktion}: Bei einer Kondensationsreaktion reagieren Moleküle unter abspaltung von Wasser oder $HCl$ miteinander. 
Man unterscheidet $Polyester$, $Polyamide$, $Polycarbonate$ und $Aminoplasten$.


\vspace{0.3cm}

\textbf{Polyester} \\
allgemein: Dicarbonsäure + Diol (Derivate)

