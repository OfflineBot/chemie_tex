
\section{Kunststoffabälle - ein Problem?}
Um einschätzen zu können wie die Produkte recycelt werden können,
müssen sie Sortenrein getrennt werden.
\begin{itemize}
    \item Schwimm-Senk-Abscheider
    \item Hydrozyklon
    \item $\implies$ Sortenreine Kunststoffe z.B. PET
\end{itemize}

\subsection{Drei Verwertungsmöglichkeiten}

\textbf{Werkstoffliches Recycling} 
\begin{itemize}
    \item ist die effizienteste Methode für Recycling von Kunststoff
    \item nur thermoplastischen Kunststoff
    \item muss sortenrein sein
    \item Kunststoffe werden zu Granulat verarbeitet
    \item meist führt die Spaltung der Makromoleküle jedoch zur Verschlechterung der Qualität (Downcycling)
    \item 41\% werden in Deutschland recycelt durch diese Methode
\end{itemize}

\vspace{0.5cm}

\textbf{Rohstoffliches Recycling} 

\begin{itemize}
    \item Monomere lassen sich selbst zurückgewinnen oder in niedermolekularen Kohlenwasserstoffe in Form von Ölen oder Gasen erzeugen
    \item \textbf{Pyrolyse}: wenn die Kunststoffe unter 700°C erhitzt werden wobei ihre Makromoleküle in kürzere Ketten zerbrechen $\implies$ Alkane, Alkene und Aromaten
    \item \textbf{Solvolyse}: Kettenteilung der Makromoleküle mithilfe eines Lösungsmittels
\end{itemize}

\vspace{0.5cm}

\textbf{Thermisches / Energetisches Recycling}

