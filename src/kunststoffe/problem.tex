
\section{Kunststoffabälle - ein Problem?}
Um einschätzen zu können wie die Produkte recycelt werden können,
müssen sie Sortenrein getrennt werden.
\begin{itemize}
    \item Schwimm-Senk-Abscheider
    \item Hydrozyklon
    \item $\implies$ Sortenreine Kunststoffe z.B. PET
\end{itemize}

\subsection{Drei Verwertungsmöglichkeiten}

\textbf{Werkstoffliches Recycling} 
\begin{itemize}
    \item ist die effizienteste Methode für Recycling von Kunststoff
    \item nur thermoplastischen Kunststoff
    \item muss sortenrein sein
    \item Kunststoffe werden zu Granulat verarbeitet
    \item meist führt die Spaltung der Makromoleküle jedoch zur Verschlechterung der Qualität (Downcycling)
    \item 41\% werden in Deutschland recycelt durch diese Methode
\end{itemize}

\vspace{0.5cm}

\textbf{Rohstoffliches Recycling} 

\begin{itemize}
    \item Ausgangsstoff: vermischte, nicht werkstofflich trennbare Kunstoffabflälle
    \item Polymere werden durch Pyrolyse oder Hydrierung in Monomere gespalten
    \item \textbf{Pyrolyse}: Erhitzung auf 700°C ohne Sauerstoff
    \item \textbf{Hydrierung}: Solvolyse; Kettenteilung durch Lösungsmittel \\
        $\rightarrow$ weitere Spaltung durch Hydrolyse mit saurer Lösung
    \item 1\% wird in Deutschland recycelt durch diese Methode
\end{itemize}

\vspace{0.5cm}

\textbf{Thermisches / Energetisches Recycling}
\begin{itemize}
    \item Verbrennung, Nutzung thermischer Energie = Erdölverbrennung
    \item Emissionen dürfen nicht überschritten werden
    \item Krise bei PVC, weil HCl Bildung
    \item Stoffe müssen rein sein, nur Kohlenwasserstoffverbindungen
    \item nach Downcycling
    \item 57\% der Kunststoffabfälle in Deutschland
\end{itemize}
