\section{Reaktionstypen}

\subsection{Polykondensations-Reaktion}

\subsection{Polyadditions-Reaktion}
Monomere mit 
\textbf{mindestens 2 funktionellen Gruppen, die mindestens eine Mehrfachbindung haben}, 
reagieren unter Übertragung eines Protons von einem Monomer zum anderen zu \textbf{Polyaddukten}. 
Es entstehen dabei keine Nebenprodukte.

Die Polyaddtion von... 
\begin{itemize}
    \item ... Monomeren mit \textbf{2 funktionellen Gruppen (=bifunktionelle Monomere)} führt zu linearen Poly\textbf{addukten} mit \textbf{thermoplastischen} Eigenschaften
    \item ... Monomeren mit \textbf{3 funktionellen Gruppen (=triefunktionelle Monomere)} führt zu vernetzten Poly\textbf{addukten} mit \textbf{duroplastischen} Eigenschaften.
\end{itemize}

\vspace{0.4cm}

\textbf{1. Polyurethan (Schäume)} \\
allgemein: Di-Isocyanat + Diol

MUSS BESSER GESCHRIEBEN WERDEN! NICHT VOLLSTÄNDIG!!

\scalebox{0.8}{
    \chemfig{
        O=C=N-R-N=C=O
    }
    +
    \chemfig{
        H-O-R-O-H
    } 
    +
    \chemfig{
        H-O-R-O-H
    }
}

\scalebox{0.8}{
    \chemfig{
        H-O-R-O^{+}(-[2]H)-C(-[6]O^{-})=N-R=N-C(-[6]O^{-})-O^{+}(-[2]H)-R-O-H
    }
}

