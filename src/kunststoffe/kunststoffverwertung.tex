
\section{Kunststoffverwertung}
\subsection{Problem}
Um einschätzen zu können wie die Produkte recycelt werden können,
müssen sie Sortenrein getrennt werden.
\begin{itemize}
    \item Schwimm-Senk-Abscheider
    \item Hydrozyklon
    \item $\implies$ Sortenreine Kunststoffe z.B. PET
\end{itemize}

\subsection{Drei Verwertungsmöglichkeiten}

\begin{multicols}{2}

    \textbf{1) Werkstoffliches Recycling} 
    \begin{itemize}
        \item ist die effizienteste Methode für Recycling von Kunststoff
        \item nur thermoplastischen Kunststoff
        \item muss sortenrein sein
        \item Kunststoffe werden zu Granulat verarbeitet
        \item meist führt die Spaltung der Makromoleküle jedoch zur Verschlechterung der Qualität (Downcycling)
        \item 41\% werden in Deutschland recycelt durch diese Methode
    \end{itemize}

\end{multicols}
\hrule
\vspace{0.3cm}
\begin{multicols}{2}

    \textbf{2) Rohstoffliches Recycling} 
    \begin{itemize}
        \item Ausgangsstoff: vermischte, nicht werkstofflich trennbare Kunstoffabflälle
        \item Polymere werden durch Pyrolyse oder Hydrierung in Monomere gespalten
        \item \textbf{Pyrolyse}: Erhitzung auf 700°C ohne Sauerstoff
        \item \textbf{Hydrierung}: Solvolyse; Kettenteilung durch Lösungsmittel \\
            $\rightarrow$ weitere Spaltung durch Hydrolyse mit saurer Lösung
        \item 1\% wird in Deutschland recycelt durch diese Methode
    \end{itemize}

\end{multicols}
\hrule
\vspace{0.3cm}
\begin{multicols}{2}

    \textbf{3) Thermisches / Energetisches Recycling}
    \begin{itemize}
        \item Verbrennung, Nutzung thermischer Energie \\ = Erdölverbrennung
        \item Emissionen dürfen nicht überschritten werden
        \item Krise bei PVC, weil HCl Bildung
        \item Stoffe müssen rein sein, \\ nur Kohlenwasserstoffverbindungen
        \item nach Downcycling
        \item 57\% der Kunststoffabfälle in Deutschland
    \end{itemize}

\end{multicols}

\subsection{Biologisch abbaubare Kunststoffe}
\textbf{Polymilchsäure (PLA)}
\begin{itemize}
    \item \textbf{Verwendung}: Verpackungsmaterial in der Lebensmittelindustrie, Mulchfolien in der Landwirtschaft,
        bioresorbierbares Nahtmaterial in der Medizintechnik
    \item \textbf{Synthese}: Polykondensation von Milchsäure-Monomeren. \\
        Ringöffnungspolymerisation von Lactid (Milchsäure-Dimeren)
    \item \textbf{Abbau}: Hydrolyse der Esterbindungen im Polymer, anschließend Abbau durch Mikroorganismen zu $CO_2$ und Wasser.
\end{itemize}

\vspace{0.3cm}

\textbf{Biologisch abbaubare Polyester (Biomüll-Folienbeutel)}
\begin{itemize}
    \item \textbf{Zusammensetzung}: Polyester, gemischt mit Maisstärke, Zellulose und PLA
    \item \textbf{Verwendung}: Kompostierbare Folienbeutel für Biomüll
\end{itemize}

\vspace{0.3cm}

\subsection{Unterschied zwischen biologisch abbaubaren- und Biokunststoffen}

\textbf{Biologisch abbaubare Kunststoffe}: Diese Kunststoffe können durch natürliche Prozesse, wie enzymatische Aktivität von Mikroorganismen (Bakterien, Pilze), in einfache Moleküle wie $CO_2$, Wasser und Biomasse zerlegt werden. 
Der Abbau erfolgt in einer ökologisch unbedenklichen Form.

\textbf{Biokunststoffe}: Biokunststoffe sind Kunststoffe, die aus nachwachsenden Rohstoffen wie Planzen (Maisstärke, Zellulose) hergestellt werden.
Sie können, müssen aber nicht biologisch abbaubar sein. Es gibt also auch Biokunststoffe, die nicht abgebaut werden, obwohl sie aus biologischen Quellen stammen.

\textbf{Fazit}: Nicht alle Biokunststoffe sind biologisch abbaubar und der Begriff "biologisch abbaubar" bezieht sich eher auf das Verhalten des Kunststoffs im Abbauprozess als auf seine Herkunft. 
Biokunststoffe beziehen sich auf die Rohstoffquelle, während biologisch abbaubare Kunststoffe die Fähigkeit haben, sich auf natürliche Weise zu zersetzen.
