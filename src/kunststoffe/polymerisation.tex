\section{Polymerisation}
Moleküle die mindestens \textbf{eine} Doppelbindung enthalten reagieren meist in einer Polymerisation.

Man unterscheidet: 
\begin{itemize}
    \item \textcolor{orange}{radikalische Polymerisation}
    \item \textcolor{cyan}{anionische Polymerisation}
    \item \textcolor{cyan}{kationische Polymerisation}
\end{itemize}

\textcolor{orange}{orange} = Wichtig fürs ABI! \\
\textcolor{cyan}{cyan} = Nicht direkt aber wichtig für Übertragung!

\subsection{Mechanismus}
\subsection{Regelung der Kettenlänge bei der Polymerisation}
über: 
\begin{itemize}
    \item Menge der Startersubstanzen [Initiator]. Je mehr desto kürzer die Kette
    \item Je höher die Temperatur $\implies$ schneller $\implies$ kurze Ketten
    \item Zugabe Inhibitoren $\implies$ Kettenwachstum wird beendet \\
        = hohe Elektronendichte
\end{itemize}

