\section{Proteine}
Reihenfolge:
\begin{enumerate}
    \item Primärstruktur: 
        Aminosäure-Sequenz (Abfolge in welcher die Aminosäuren verbunden sind) \\
        mit C-terminalen Ende + N-terminalen Ende
    \item Sekundärstruktur: \\
        \textrightarrow\ $\alpha$-Helix (intramolekulare H-Brücken-Wecheslwirkungen)\\
        \textrightarrow\ $\beta$-Faltblattstruktur (intermolekulare H-Brücken-Wechselwirkungen)
    \item Tertiärstruktur: \\
        meiste Proteine enden hier. \\
        räumliche Bau des Protein: \\
        \textrightarrow\ Globuläres Protein $|$ Enzyme (kugelförmig) \\
        \textrightarrow\ Fibrilläres Protein $|$ Strukturproteine (faserförmig)

        Wecheslwirkungen: 
        \begin{itemize}
            \item London-Wechselwirkungen
            \item Dipol-Dipol-Wechselwirkungen
            \item H-Brücken-Wecheslwirkungen
            \item Ionische-Wechselwirkungen = (Ionische-Bindung)
            \item Atombindung
        \end{itemize}
    \item Quartärstruktur: \\
        Mehrere globuläre Proteine verbinden sich zu einer größeren Einheit
\end{enumerate}


\subsection{Denaturierung von Proteinen}
Zerstörung von Secundär- und Tertiärstruktur
\begin{itemize}
    \item Wärmezufuhr
    \item Säure-/Base Zufuhr (pH-Wert)
    \item Schwermetalle (Blei, Kupfer, etc.)
\end{itemize}


\subsection{Protein Enzyme}
Proteine können als Enzyme fungieren. \\
Enzyme sind Biokatalysatoren die nach dem Schlüssel-Schloss Prinzip wirken.
