
\section{Begriffe} \label{sec:begriffe}


\subsection{Chiralität} \label{sec:chiral}
Bild und Spiegelbild lassen sicht zur Deckung bringen. \\
Chirale Verbindungen sind optisch aktiv 

(GUTE ERKLÄRUNG FEHLT!)

\subsection{Achiralität}
Bild und Spiegelbild lassen sich zur Deckung bringen. (vergleiche mit \hyperref[sec:chiral]{Chiralität})

\subsection{Isomerie}
Stoffe mit der selben Summenformel aber unterschiedlicher Strukturformel. (Andere räumliche Struktur)
\begin{enumerate}
    \item Konstitutionsisomere: \\
        Beispiel: 2-Methylpropan und Butan
    \item Stereoisomere \\
        Unterscheiden sich nur in der räumlichen Anordnung
    \item Enantionmere \\
        Sind Stereoisomere; Spiegelbild aber nicht identisch
    \item Diastomere
        sind Stereoisomiere; Keine Enantiomere \textrightarrow\ nicht im Spiegelbild gleich
\end{enumerate}

\subsection{Anomere}
Eine Art der Isomerie. 
Wird nur an dem anomeren Zentrum (erste C-Atom von Haworth) unterschieden.
Entweder $\alpha$ oder $\beta$. 

Bei $\alpha$: Hydroxylgruppe geht nach unten.\\
Bei $\beta$: Hydroxylgruppe geht nach oben.

\subsection{Asymmetrisches C-Atom}
Ein C-Atom bei dem alle "Ärmchen" unterschiedliche Bindungen haben. \\
Beispiel:
\begin{enumerate}
    \item - H
    \item - OH
    \item - Methan
    \item - Ethan
\end{enumerate}

