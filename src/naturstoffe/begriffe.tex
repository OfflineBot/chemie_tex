
\section{Begriffe} \label{sec:begriffe}
\subsection{Isomerie}
Stoffe mit der selben Summenformel aber unterschiedlicher Strukturformel. (Andere räumliche Struktur)
\begin{enumerate}
    \item Konstitutionsisomere: \\
        Beispiel: 2-Methylpropan und Butan

    \item Stereoisomere \\
        Unterscheiden sich nur in der räumlichen Anordnung
    \item Enantionmere \\
        Sind Stereoisomere; Spiegelbild aber nicht identisch
    \item Diastomere
        sind Stereoisomiere; Keine Enantiomere \textrightarrow\ nicht im Spiegelbild gleich

\end{enumerate}
\subsection{Asymmetrisches C-Atom}
Ein C-Atom bei dem alle "Ärmchen" unterschiedliche Bindungen haben. \\
Beispiel:
\begin{enumerate}
    \item - H
    \item - OH
    \item - Methan
    \item - Ethan
\end{enumerate}

\subsection{Chiralität}
\subsection{Alpha-Beta}

\subsection{Fehling/Tollens Reaktion}
