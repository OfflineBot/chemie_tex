
\section{Begriffe} \label{sec:begriffe}


\subsection{Chiralität} \label{sec:chiral}
\begin{itemize}
    \item verhält sich wie Bild und Spiegelbild
    \item optisch Aktiv \\
        \textrightarrow\ fähigkeit bei polarisierendem Licht zu drehen
\end{itemize}

Beispiel:
Hände. Sind gleich aber nicht identisch. Spiegelbild ist identisch zur Hand.

\subsection{Achiralität}
Bild und Spiegelbild lassen sich nicht zur Deckung bringen. Besitzen eine Symmetrieebene. 

Beispiel:
$CO_2$ \\
$O = C = O$

\subsection{Isomerie}
\label{sec:isomere}
Stoffe mit der selben Summenformel aber unterschiedlicher Strukturformel. \\
(Andere räumliche Struktur/Aufbau) \\
Folgende Isomere: 
\begin{enumerate}
    \item Stereoisomere:
        \begin{itemize}
            \item unterscheiden sich nur in der räumlichen Anordnung
            \item haben selbe Anordnung von Atomen
        \end{itemize}
    \item Konstitutionsisomere:
        \begin{itemize}
            \item unterschiedliche Bindungen zwischen Atomen \\
                \textrightarrow\ Atome sind in unterschiedlicher reihenfolge verbunden
            \item Beispiel: \\
                $C_4H_10$ könnte Butan und 2-Methylpropan sein
        \end{itemize}
        Beispiel: 2-Methylpropan und Butan
    \item Enantionmere:
        \begin{itemize}
            \item gehören zu Stereoisomeren
            \item nicht deckungsgleich (wie Hände)
            \item gleiche Bindungen
            \item wie Bild und Spiegelbild
            \item ist \hyperref[sec:chiral]{Chiral}
            \item kann mit \hyperref[sec:fischer]{Fischerprojektion} erfasst werden.
        \end{itemize}
    \item Diastomere:
        \begin{itemize}
            \item gehören zu Stereoisomeren
            \item kann mit \hyperref[sec:fischer]{Fischerprojektion} erfasst werden.
            \item keine Enantiomere da sie nicht im Spiegelbild gleich sind.
        \end{itemize}
\end{enumerate}

\subsection{Anomere}
Eine Art der Isomerie. 
Wird nur an dem anomeren Zentrum (erste C-Atom von Haworth) unterschieden.
Entweder $\alpha$ oder $\beta$. 

Bei $\alpha$: Hydroxylgruppe geht nach unten.\\
Bei $\beta$: Hydroxylgruppe geht nach oben.


\subsection{Asymmetrisches C-Atom}
Auch "Chirales Zentrum" genannt. \\
Ein C-Atom bei dem alle "Ärmchen" unterschiedliche Bindungen haben. Dieser werden mit einem * markiert.

Beispiel:
\begin{enumerate}
    \item - H
    \item - OH
    \item - Methan
    \item - Ethan
\end{enumerate}

\subsection{Furanose, Pyranose}
Die Begriffe beschreiben den Ring eines Stoffes. \\
Furanose steht für einen Fünf-Ring (5-C-Atome) \\
Pyranose steht für einen Sechs-Ring (6-C-Atome)
