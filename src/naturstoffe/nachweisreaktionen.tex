\section{Nachweisreaktionen}
Meist Reaktionen bei denen die Farben der Endstoffe der Nachweis ist.

\subsection{Stärke}
\label{sec:nachweis_stärke}
Iod-Probe:
\begin{enumerate}
    \item Probe vorbereitung: \\
        Die zu testende Substanz in Wasser lösen
    \item Iodlösung hinzufügen: \\
        Einige Tropfen einer Iod-Kaliumiodid-Lösung hinzufügen
    \item Farbänderung: \\
        Bei Anwesenheit von Stärke färbt sich die Lösung blau-schwarz
\end{enumerate}

\subsection{Aminosäuren}
\label{sec:nachweis_amino}

\subsection{Proteine}
\label{sec:nachweis_proteine}
Xanto-Protein-Reaktion \\
Vorraussetzungen:
\begin{itemize}
    \item anwesenheit von Aminosäure mit Ring
\end{itemize}
Funktionsweise: 
\begin{itemize}
    \item Bei beigabe von Salpetersäure färbt sich die Substanz gelb
\end{itemize}
