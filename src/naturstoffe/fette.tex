\section{Stoffklasse Fette}
Fettmoleküle mit (vielen) \underline{ungesättigten} Fettsäuren sind meistens Öle. \\
Fettmoleküle mit (vielen) \underline{gesättigten} Fettsäuren sind bei Raumtemperatur meistens fest. \\
$\Rightarrow$ Fette besitzen einen Schmelzbereich 

Je langkettiger die Fettsäurereste sind, desto höher liegt der Schmelzbereich des Fettes. \\
Je mehr C=C (Doppelbindungen) vorkommen, desto niedriger ist der Schmelzbereich des Fettes.

\subsection{Chemischer Aufbau von Fetten und Ölen}
Fettmoleküle sind Ester des Glycerins (1,2,3-Propandiol) mit drei Fettsäuren.

Glycerin:\\
\scalebox{0.8}{
\chemfig{
    C(-[2]H)(-[4]H)(-[8]OH)
    -[6]C(-[4]H)(-[8]OH)
    -[6]C(-[6]H)(-[4]H)(-[8]OH)
}
}

Fettsäuren gibt es beliebige. Hier:\\
\scalebox{0.8}{
\chemfig{
    C(-[4]O(-[4]H))(=[2]O)-C-Rest
}
}

Wird zu Fett: ($H_2O$ wird abgespaltet) \\
\scalebox{0.8}{
\chemfig{
    C(-[2]H)(-[4]H)(-[8]O-C(=[2,0.7]O)(-[8]C-Rest))
    -[6,1.5]C(-[4]H)(-[8]O-C(=[2,0.7]O)(-[8]C-Rest))
    -[6,1.5]C(-[6]H)(-[4]H)(-[8]O-C(=[2,0.7]O)(-[8]C-Rest))
}
}

\subsection{Fetthärtung/Margarine}
Pflanzenöle wie zum Beispiel Ölsäure werden mit Wasserstoff ($H_2$) und 
Nickel\hyperref[sec:katalysator]{katalysatoren}
gehärtet

$\rightarrow$ Additionsreaktion

\subsection{Begriffe}
\begin{itemize}
    \item Gesättigte Fettsäuren: Fettsäuren ohne C-Doppelbindungen
    \item Ungesättigte Fettsäuren: Fettsäuren mit C-Doppelbindung
    \item Essenzielle Fettsäuren: vom Körper nicht herstellbar, muss über Nahrung aufgenommen werden.
\end{itemize}

\subsection{Ungesättigte Fettsäuren Nachweis}
\label{sec:nachweis_ung_fettsäuren}
Nachweis der Doppelbindung C=C
\begin{itemize}
    \item Bromwasser ($Br_{2\ (aq}$)
    \item $KMnO_4$ Bayer-Reagenz
    \item Iod $I_2$ $\Rightarrow$ Iodzahl zu bestimmen
\end{itemize}
